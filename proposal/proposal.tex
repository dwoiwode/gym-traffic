\documentclass[11pt]{scrartcl}

% Packages
\usepackage{amsmath}
\usepackage{xcolor}
\usepackage{multicol}
\usepackage{enumitem}
\usepackage[bottom=1cm]{geometry}

\makeatletter
\usepackage[
%    pdftitle={\@title},
%    pdfauthor={\@author},
%    pdfsubject={Reiforcement Learning}
]{hyperref}
\makeatother

% Style settings
\setlength{\parindent}{0}
\RedeclareSectionCommand[
	beforeskip=6pt,
	afterskip=6pt
]{section}

% Settings
\title{Env: Traffic Lights Simulation}
\author{Dominik Woiwode}




% Document
\begin{document}
    \begin{center}
        \makeatletter
        {\Huge\bfseries\@title}\\
        {\large\bfseries\@author \quad-\quad\@date}
        \makeatother
    \end{center}


    \section{Description}\label{sec:description}
    This environment tries to simulate a network of streets, vehicles and intersections with traffic lights.
    The agent is responsible for handling the traffic lights at the intersections in a way that the achieved traffic flow is maximized.
    Furthermore all vehicles have to reach its goal in a given time.
    The traffic flow will be calculated by the amount of vehicles moving.

    Each vehicle follows a given path and from a starting position and tries to reach its destination.
    The path is precomputed with a simple search-algorithm and vehicles are not allowed to touch/crash into each other.

    First, a simple simulation based on discrete values will be implemented.
    This environment can be extended further to increase the complexity:
    \begin{itemize}[noitemsep]
        \item by using continuous values for vehicle positions
        \item by using continuous values for the street network
        \item by improving vehicle ai (e.g. lane switching, acceleration, rerouting based on traffic, etc.)
    \end{itemize}

    The number of vehicles in front of a traffic light can be used as observation-space.
    This might be interpreted as a camera in the real world, which can identify and count waiting vehicles.

    In a more complex network with two or more intersections it has to be considered whether each intersection is handled by a different agent or whether one agent handles all traffic lights.


    \section{Frameworks}
    Python 3.8.6 will be used for this project, as this version of Python was already used for all the exercises.
    It is currently not planed to use any framework for the environment besides \href{https://gym.openai.com/}{OpenAI-Gym} and maybe \href{https://numpy.org/}{numpy}.
    For the agents it is planned to use recommended agents provided by \href{https://github.com/hill-a/stable-baselines}{hill-a/stable-baselines}.


    \section{Computing Resources}
    For training the agent I can use a Laptop with an Nvidia GeForce 940MX.
    It will be checked whether I need to use Google Collab.


    \section{Timeline}
    These are the milestones planned for this project. All dates are given for the year 2021.
    \begin{itemize}[noitemsep]
        \item[08/02] Working simple environment (e.g. a single plusshaped crossing with traffic lights)
        \item[29/02] More complex environment with at least one working agent
        \item[10/03] Comparison of different agents solving environments with different difficulty levels \\
                     Start working on presentation slides
        \item[17/03] Deadline for submission
    \end{itemize}


%    \clearpage
%    \part*{Appendix}
%
%
%    \section{Features}\label{subsec:features}
%    \newcommand{\dwyes}[1]{\textcolor{black}{#1}}
%    \newcommand{\dwmaybe}[1]{\textcolor{darkgray}{(#1)}}
%    \newcommand{\dwno}[1]{\textcolor{gray}{#1}}
%
%    Here is an overview of features that shape the type of this environment:
%    \begin{center}
%        \begin{tabular}{|c|c|p{0.5\linewidth}|}
%            \hline
%            \multicolumn{2}{|c|}{\textbf{Features}} & \textbf{Comment} \\
%            \hline
%            \hline
%            \dwyes{Fully Observable} & \dwyes{Partially Observable} & Depending on implementation                                                            \\
%            \hline
%            \dwno{Static}            & \dwyes{Dynamic}              & The simulation continues even if the agent does nothing                                \\
%            \hline
%            \dwno{Discrete}          & \dwyes{Continuous}           & The positions of the cars will be in continuous space                                  \\
%            \hline
%            \dwyes{Deterministic}    & \dwno{Stochastic}            & The simulation itself is deterministic. The spawning of cars could be randomized       \\
%            \hline
%            \dwyes{Single Agent}     & \dwyes{Multi Agent}          & There could be an agent for each crosssection or a single agent for all traffic lights \\
%            \hline
%            \dwno{Competitive}       & \dwyes{Collaborative}        & Assuming multiple agents, they would try to help each other                            \\
%            \hline
%            Episodic                 & Sequential                   &                                                                                        \\
%            \hline
%        \end{tabular}
%    \end{center}

\end{document}